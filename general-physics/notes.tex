\documentclass[12pt,a4paper]{report}

% math packages
\usepackage{amsmath,amssymb,amsthm}

% fonts
\usepackage{fontspec}
\renewcommand{\familydefault}{\sfdefault}
\setsansfont[
  BoldFont={WorkSans-Bold.ttf}, 
  ItalicFont={WorkSans-Italic.ttf},
  BoldItalicFont={WorkSans-BoldItalic.ttf}
  ]{WorkSans-Regular.ttf}
\setmonofont{Source Code Pro}
\usepackage[no-math]{luatexja-fontspec}
\setmainjfont[BoldFont=Noto Serif TC Bold]{Noto Serif TC}
\setsansjfont[BoldFont=Noto Sans TC Bold]{Noto Sans TC}
\ltjsetparameter{jacharrange={-1, -2, +3, -4, -5, +6, +7, +8}}
\setmonofont{Monaco}

% funny emoji
\usepackage{emoji}
\setemojifont{Twemoji Mozilla}

% headers and footers
\usepackage[margin=1.5cm]{geometry}
\addtolength{\headheight}{1cm}
\addtolength{\textheight}{-1cm}
\usepackage{fancyhdr}
\pagestyle{fancy}
\renewcommand{\headrulewidth}{0pt}
\usepackage{calc}
\fancypagestyle{firststyle}
{
\rhead{\makebox[3cm][r]{
  \raisebox{-\totalheight+0.6cm}[0pt][0pt]{
    \includegraphics[width=3cm]{qr}\hspace{-1em}}}}
    }
\cfoot{\thepage}

% paragraph formatting
\usepackage{indentfirst}
\setlength{\parindent}{2em}
\setlength{\parskip}{0.5em}
\renewcommand{\baselinestretch}{1.25}
\usepackage{setspace}

% tabular
\usepackage{ctable}
\usepackage{tabularx}

% math convensional macro
\newcommand{\ve}{\varepsilon}
\newcommand{\overbar}[1]{\mkern1.5mu\overline{\mkern-1.5mu#1\mkern-1.5mu}\mkern1.5mu}
\renewcommand{\qedsymbol}{\(\blacksquare\)}


\begin{document}
  \title{General Physics Note}
  \author{LittleCube \Huge\emoji{ice-cube}}

  \maketitle

  \section*{3 Vectors}
  Assume two vectors \(\vec v_1,\vec v_2\) are differ by \(\theta\) radius, then the magnitude of \(\vec v_1 + \vec v_2\) is  
  \[\sqrt{ |\vec v_1|^2 + |\vec v_2|^2 - 2|\vec v_1| |\vec v_2| \cos \boldsymbol{(\pi - \theta)}}.\]
  \textbf{It is better to decompose them along the axes and add them up.}

  \section*{4 Motion in Two and Three Dimensions}
  \subsection*{Proof of Uniform Circular Motion Acceleration}
  A particle in two dimension is moving countercolockwise in a circular motion revolving the origin with radius \(R\), its velocity is constant and tangent to the circle, therefore
  \[v_x = - v \sin \theta, v_y = v \cos \theta.\]
  Substituting \(\cos \theta, \sin \theta\) with displacement gives
  \[v_x = - v \cdot \frac{x_y}{R}, v_y = v \cdot \frac{x_x}{R}.\]
  Differentiating them yields the acceleration,
  \[a_x = - v \cdot \frac{v_y}{R} = - v^2 \cdot \frac{\sin \theta}{R}, a_y = v \cdot \frac{v_x}{R} = - v^2 \cdot \frac{\cos \theta}{R}.\]
  Therefore the magnitude of the acceleration is
  \[a = \frac{v^2}{R}\]

  \section*{5 Force and Motion -- I}
  \subsection*{A Confusion I Have Made For a Long Time}

  The unit \textbf{Newton} is defined by
  \[1 \mathrm{N} = 1 \mathrm{kg \cdot m/s^2}\]
  Therefore, an object with mass \(m\) has the weight \(mg\) where \(g\) is the gravitational acceleration at the place.  
  1 \(\mathrm{N}\) is \textbf{not} defined as 9.8 \(\mathrm{kg}\). It is a \textbf{misunderstand} (since in most cases we are dealing with objects on the Earth).
  
  \subsection*{Drag Force}
  The formula to the drag force is
  \[\frac 1 2 C \rho A v^2,\]
  where \(C\) is the \textbf{drag coefficient}, \(\rho\) is the \textbf{density}, \(A\) is the \textbf{effective cross-section area} (the area that is facing toward the front; the area that is perpendicular to the direction of velocity).

  \subsection*{Centripedal Force Does Not Change the Speed}
  Let's say we have two vector functions: velocity \(\mathbf v(t)\) and acceleration \(\mathbf a(t)\) which satisfy \(\langle \mathbf v(t), \mathbf a(t)\rangle = 0\). We want to show that
  \[\|\mathbf v(t)\| = \text{constant}.\]
  
  \begin{proof}
    We know that
    \[\langle \mathbf v(t), \mathbf v(t)\rangle = \|v(t)\|^2.\]
    Differentiating both sides gives
    \[\langle \mathbf a(t), \mathbf v(t)\rangle + \langle \mathbf v(t), \mathbf a(t)\rangle = \frac{d}{dt}\|v(t)\|^2\]
    \[2\langle \mathbf a(t), \mathbf v(t)\rangle = 0 = \frac{d}{dt}\|v(t)\|^2\]
    Therefore \(\|v(t)\|\) is constant. 
  \end{proof}

  \section*{8 Potential Energy and Conservation of Energy}
  The mechanical energy \(E_{mec}\) of a system can only inclue 
  \begin{itemize}
    \item Kinetic energy \(K\)  
    \item Potential energy \textbf{between objects inside the system}
  \end{itemize}
  
  For example, a ball is free falling. If the system contains \textbf{only the ball}, then it has kinetic energy of \(K = \dfrac 1 2 mv^2\) and a \textbf{constant} potential energy \(U\). The Earth is constantly exerting force on the system, causing the mechanical energy to increase.  

\end{document}
