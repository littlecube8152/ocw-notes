\documentclass[12pt,a4paper]{article}

% math packages
\usepackage{amsmath,amssymb,amsthm}

% fonts

\usepackage{fontspec}
\renewcommand{\familydefault}{\sfdefault}
\setsansfont{Work Sans}[
  Renderer = HarfBuzz,
]
\setmonofont{SourceCodePro}
\usepackage[no-math]{luatexja-fontspec}
\setmainjfont[BoldFont=Noto Sans CJK TC Bold]{Noto Sans CJK TC}
\setsansjfont[BoldFont=Noto Sans CJK TC Bold]{Noto Sans CJK TC}
% \ltjsetparameter{jacharrange={-1, -2, +3, -4, -5, +6, +7, +8}}
% \setmonofont{Monaco}

% funny emoji
\usepackage{emoji}
\setemojifont{Twemoji Mozilla}

% headers and footers
\usepackage[margin=1.5cm]{geometry}
\addtolength{\headheight}{1cm}
\addtolength{\textheight}{-1cm}
\usepackage{fancyhdr}
\pagestyle{fancy}
\renewcommand{\headrulewidth}{0pt}
\usepackage{calc}


% paragraph formatting
\usepackage{indentfirst}
\setlength{\parindent}{2em}
\setlength{\parskip}{0.5em}
\renewcommand{\baselinestretch}{1.25}
\usepackage{setspace}

% tabular
\usepackage{ctable}
\usepackage{tabularx}

% math convensional macro
\newcommand{\ve}{\varepsilon}
\newcommand{\overbar}[1]{\mkern1.5mu\overline{\mkern-1.5mu#1\mkern-1.5mu}\mkern1.5mu}
\renewcommand{\qedsymbol}{\(\blacksquare\)}


\begin{document}
  \title{\vspace{-4em}Introduction to Computers \textbf{HW2}}
  \author{B12902057 王淇}
  \date{}
  \maketitle
  % \section*{Problem Solutions}
  \begin{enumerate}
    \item 
      \begin{enumerate}
        \item \texttt{127}
        \item \texttt{-75}
        \item \texttt{-42}
        \item \texttt{-18}
        \item \texttt{-124}
      \end{enumerate}
      \item 
      \begin{enumerate}
        \item \texttt{0111100001011010}
        \item \texttt{101100001000}
        \item \texttt{1000101110011110}
        \item \texttt{110101001001}
        \item \texttt{110110110011}
      \end{enumerate}
      \item \texttt{ltz E@sy haH@}
      \item 
      \begin{enumerate}
        \item $\mathbf 1 \cdot 2^5 + 
        \mathbf 0 \cdot 2^4 + 
        \mathbf 1 \cdot 2^3 + 
        \mathbf 0 \cdot 2^2 + 
        \mathbf 1 \cdot 2^1 + 
        \mathbf 1 \cdot 2^0 = 43$
        \item $\mathbf 1 \cdot 2^4 + 
        \mathbf 0 \cdot 2^3 + 
        \mathbf 0 \cdot 2^2 + 
        \mathbf 1 \cdot 2^1 + 
        \mathbf 1 \cdot 2^0 + 
        \mathbf 0 \cdot 2^{-1} + 
        \mathbf 1 \cdot 2^{-2} + 
        \mathbf 1 \cdot 2^{-3} = 19.375$
        \item $\mathbf 2 \cdot 8^1 + 
        \mathbf 4 \cdot 8^0 + 
        \mathbf 3 \cdot 8^{-1} = 20.375$
        \item $\mathbf 1 \cdot 16^1 + 
        \mathbf {10} \cdot 16^0 + 
        \mathbf {14} \cdot 16^{-1} = 26.875$
        \item $\mathbf 5 \cdot 16^2 + 
        \mathbf {15} \cdot 16^1 + 
        \mathbf {11} \cdot 16^0 = 1531$
      \end{enumerate}
      \item 
      \begin{enumerate}
        \item (1)0000
        \item (1)0011
        \item 1110
        \item (1)0010
        \item None of them causes any overflow.
      \end{enumerate}
      \item 
      \begin{enumerate}
        \item False. Consider $-1 + -1$ always cause a carry at  33-rd bit but the result is correct ($-2$).
        \item True.
      \end{enumerate}
      \item 
      \begin{enumerate}
        \item $3 \times 60 \times 44100 \times 16 = 127008000 \texttt{(bits)} = 15503.90625 \texttt{(KB)}$
        \item $10 \times 1920 \times 1080 \times 8 \times 3 \times 60 = 29859840000 \texttt{(bits)} \approx 3559.570312 \texttt{(MB)}$
      \end{enumerate}
      \item 
      \begin{enumerate}
        \item \texttt{10$_{16}$}
        \item \texttt{3B$_{16}$}
        \item \texttt{E9$_{16}$}
        \item \texttt{10$_{16}$}
        \item \texttt{3C$_{16}$}. Only this summation causes overflow since the result is \texttt{13C$_{16}$} which is too large to fit in a 8-bit unsigned number.
      \end{enumerate}
      \item 
      \texttt{
        \begin{tabular}[t]{c|c|c|c}
          A & B & C & result \\
          \hline
          0 & 0 & 0 & 0 \\
          0 & 0 & 1 & 0 \\
          0 & 1 & 0 & 1 \\
          0 & 1 & 1 & 0 \\
          1 & 0 & 0 & 1 \\
          1 & 0 & 1 & 1 \\
          1 & 1 & 0 & 0 \\
          1 & 1 & 1 & 0 \\
        \end{tabular}
      } 
  \end{enumerate}
\end{document}
