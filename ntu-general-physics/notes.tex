\documentclass[12pt,a4paper]{report}

% math packages
\usepackage{amsmath,amssymb,amsthm}

% fonts
\usepackage{fontspec}
\renewcommand{\familydefault}{\sfdefault}
\setsansfont[
  BoldFont={WorkSans-Bold.ttf}, 
  ItalicFont={WorkSans-Italic.ttf},
  BoldItalicFont={WorkSans-BoldItalic.ttf}
  ]{WorkSans-Regular.ttf}
\setmonofont{Source Code Pro}
\usepackage[no-math]{luatexja-fontspec}
\setmainjfont[BoldFont=Noto Serif CJK TC Bold]{Noto Serif CJK TC}
\setsansjfont[BoldFont=Noto Sans CJK TC Bold]{Noto Sans CJK TC}
\ltjsetparameter{jacharrange={-1, -2, +3, -4, -5, +6, +7, +8}}
% \setmonofont{Monaco}

% funny emoji
\usepackage{emoji}
\setemojifont{Twemoji Mozilla}

% headers and footers
\usepackage[margin=1.5cm]{geometry}
\addtolength{\headheight}{1cm}
\addtolength{\textheight}{-1cm}
\usepackage{fancyhdr}
\pagestyle{fancy}
\renewcommand{\headrulewidth}{0pt}
\usepackage{calc}
\fancypagestyle{firststyle}
{
\rhead{\makebox[3cm][r]{
  \raisebox{-\totalheight+0.6cm}[0pt][0pt]{
    \includegraphics[width=3cm]{qr}\hspace{-1em}}}}
    }
\cfoot{\thepage}

% paragraph formatting
\usepackage{indentfirst}
\setlength{\parindent}{2em}
\setlength{\parskip}{0.5em}
\renewcommand{\baselinestretch}{1.25}
\usepackage{setspace}

% tabular
\usepackage{ctable}
\usepackage{tabularx}

% math convensional macro
\newcommand{\ve}{\varepsilon}
\newcommand{\overbar}[1]{\mkern1.5mu\overline{\mkern-1.5mu#1\mkern-1.5mu}\mkern1.5mu}
\renewcommand{\qedsymbol}{\(\blacksquare\)}


\begin{document}
  \title{General Physics Note}
  \author{LittleCube \Huge\emoji{ice-cube}}

  \maketitle

  \section*{3 Vectors}
  Assume two vectors \(\vec v_1,\vec v_2\) are differ by \(\theta\) radius, then the magnitude of \(\vec v_1 + \vec v_2\) is  
  \[\sqrt{ |\vec v_1|^2 + |\vec v_2|^2 - 2|\vec v_1| |\vec v_2| \cos \boldsymbol{(\pi - \theta)}}.\]
  \textbf{It is better to decompose them along the axes and add them up.}

  % \section*{4 Motion in Two and Three Dimensions}
  % \subsection*{Proof of Uniform Circular Motion Acceleration}
  % A particle in two dimension is moving countercolockwise in a circular motion revolving the origin with radius \(R\), its velocity is constant and tangent to the circle, therefore
  % \[v_x = - v \sin \theta, v_y = v \cos \theta.\]
  % Substituting \(\cos \theta, \sin \theta\) with displacement gives
  % \[v_x = - v \cdot \frac{x_y}{R}, v_y = v \cdot \frac{x_x}{R}.\]
  % Differentiating them yields the acceleration,
  % \[a_x = - v \cdot \frac{v_y}{R} = - v^2 \cdot \frac{\sin \theta}{R}, a_y = v \cdot \frac{v_x}{R} = - v^2 \cdot \frac{\cos \theta}{R}.\]
  % Therefore the magnitude of the acceleration is
  % \[a = \frac{v^2}{R}\]

  \section*{5 Force and Motion -- I}
  \subsection*{A Confusion I Have Made For a Long Time}

  The unit \textbf{Newton} is defined by
  \[1 \mathrm{N} = 1 \mathrm{kg \cdot m/s^2}\]
  
  \subsection*{Drag Force}
  The formula to the drag force is
  \[\frac 1 2 C \rho A v^2,\]
  where \(C\) is the \textbf{drag coefficient}, \(\rho\) is the \textbf{density}, \(A\) is the \textbf{effective cross-section area} (the area that is facing toward the front; the area that is perpendicular to the direction of velocity).

  \subsection*{Centripedal Force Does Not Change the Speed}
  Let's say we have two vector functions: velocity \(\mathbf v(t)\) and acceleration \(\mathbf a(t)\) which satisfy \(\langle \mathbf v(t), \mathbf a(t)\rangle = 0\). We want to show that
  \[\|\mathbf v(t)\| = \text{constant}.\]
  
  \begin{proof}
    We know that
    \[\langle \mathbf v(t), \mathbf v(t)\rangle = \|v(t)\|^2.\]
    Differentiating both sides gives
    \[\langle \mathbf a(t), \mathbf v(t)\rangle + \langle \mathbf v(t), \mathbf a(t)\rangle = \frac{d}{dt}\|v(t)\|^2\]
    \[2\langle \mathbf a(t), \mathbf v(t)\rangle = 0 = \frac{d}{dt}\|v(t)\|^2\]
    Therefore \(\|v(t)\|\) is constant. 
  \end{proof}

  \section*{8 Potential Energy and Conservation of Energy}
  The mechanical energy \(E_{mec}\) of a system can only inclue 
  \begin{itemize}
    \item Kinetic energy \(K\)  
    \item Potential energy \textbf{between objects inside the system}
  \end{itemize}
  
  For example, a ball is free falling. If the system contains \textbf{only the ball}, then it has kinetic energy of \(K = \dfrac 1 2 mv^2\) and a \textbf{constant} potential energy \(U\). The Earth is constantly exerting force on the system, causing the mechanical energy to increase.  

  \section*{37 Relativity}
  For special relativity, all discussions are based on inertial reference frames (i.e. we don't take acceleration into consideration).

  Special relativity is based on two postulates: 
  \begin{enumerate}
    \item \textbf{Laws of physics} stay the same.
    \item The \textbf{speed of light} stays the same, $c = 299\ 792\ 458\ \mathrm{m/s}$.
  \end{enumerate}
  
  \subsection*{Time Dilation}
  When two events occur at the \textbf{same location} in an inertial reference frame, the time interval between them measured in that frame is called the \textbf{proper time} $\Delta t_0$.  

  For any other reference frame move at speed $v$ relative to that frame, the interval between the two event will be measured as

  \[ \Delta t = \gamma \Delta t_0, \text{where }\gamma = \frac{1}{1 - \beta^2} \text{ is called \textbf{Lorentz factor} with } \beta = \frac v c \]

  \subsection*{Length Contraction}
  The length $L_0$ of an object measured in the rest frame of the object is its \textbf{proper length} or rest length.

  For any other frame move at speed $v$ relative to that frame, the object length will be measured as 

  \[ L = \frac {L_0} \gamma \]

  Length contraction only occurs along the direction of the relative speed.  

  \subsection*{Lorentz Transformation}
  A event $E$ measured by two different inertial frame $F$ and $F'$, where $F'$ moves at speed $v$ along their common $x$ axes, has different space-time coordinates. This pair of coordinates can be transformed as: 

  \[ \begin{matrix} x \\ t \end{matrix} \longrightarrow \begin{matrix} x' & = & \gamma (x - vt) \\ t' & = & \gamma \left(t - \dfrac{vx}{c^2}\right) \end{matrix} \]

  It can be derived from time dilation and length contraction.

  For two different event, with different $x$ and $t$, let $\Delta x, \Delta t$ be their x-coordinate and time difference, we shift the coordinates from $F'$ to $F$:   

  \[ \begin{matrix} \Delta x' & = & \gamma (\Delta x - v\Delta t) \\ \Delta t' & = & \gamma \left(\Delta t + \dfrac{v\Delta x}{c^2}\right) \end{matrix} \Rightarrow v' = \frac{\Delta x'}{\Delta t'} = \frac{u + v}{1 + \frac{uv}{c^2}}, \text{where } u = \frac{\Delta x}{\Delta t} \]

  From which we derived the formula of adding velocities. Note when one of $u, v$ is $c$, the result is always $c$.

  \subsection*{Doppler Effect for Light}
  Assume the source and detector are moving \textbf{toward} each other with velocity $v$. Let $f_0$ be the \textbf{proper frequency} (frequency measured by the source) and $\lambda_0$ be the correspond \textbf{proper wavelength}. To the source, the time duration between two consecutive light wave arrives to the detector $t_s$ is
  
  \[ \lambda_0 = ct_s - vt_s \Rightarrow t_s = \frac{\lambda_0}{c-v} = \frac 1 {f_0 (1 - \beta)}\]

  However, the time is actually \textit{dilated}. The actual frequency and wavelength observed by the detector is

  \[ f = \frac 1 t = \frac{\gamma}{t_s} = \frac{f_0(1 - \beta)}{\sqrt{1 - \beta^2}} = f_0 \sqrt \frac{1 - \beta}{1 + \beta} \Rightarrow \lambda = \lambda_0 \sqrt \frac{1 + \beta}{1 - \beta} \]

  \subsection*{Momentum and Energy}

  We start with the new definition:  

  \[ p = mv = m \frac{\Delta x}{\Delta t_0} \]

  Since the observer's time is \textit{dilated}, so our observed time is actually longer, therefore 

  \[ p = m \frac{\Delta x}{\Delta t_0} = m \frac{\Delta x}{\Delta t} \cdot \gamma = \gamma mv \]
\end{document}
